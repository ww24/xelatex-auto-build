\documentclass[a4paper]{article}
\title{\XeLaTeX Auto Build}
\author{Takenori Nakagawa}
\date{2013/10/16}

% \usepackage{listings,jlisting}
% \lstset{
%   basicstyle=\ttfamily\scriptsize,
%   frame=single
% }

% \usepackage{here}

% \usepackage{geometry}
% \usepackage[showframe]{geometry}
% \geometry{
%   top=25mm,
%   left=20mm,
%   right=15mm,
%   bottom=15mm
% }

% \usepackage{titling}
% \setlength{\droptitle}{-14mm}

\usepackage{fontspec}
\usepackage{xltxtra}
\usepackage{zxjatype}
\setjamainfont{ipaexm.ttf}
\setjasansfont{ipaexg.ttf}
\setjamonofont{ipaexg.ttf}
\XeTeXlinebreaklocale "ja"

\begin{document}
\maketitle
% \fontsize{11pt}{15pt}\selectfont

\begin{abstract}
\center
This provides an environment to automatically build \XeLaTeX by Node.js.
\end{abstract}

\section{Features}
\begin{itemize}
  \item Generate from LaTeX to PDF (require xelatex command)
  \item HTTP Server \& Livereload
\end{itemize}

\section{How to Start}
\begin{itemize}
  \item {\tt npm install}
  \item {\tt npm start}
  \item save {\tt src/report.tex}
  \item open {\tt http://localhost:8000/report.pdf}
\end{itemize}

\section{Directory}
\begin{description}
  \item[dest]\mbox{}\\
  Directory where the PDF will be outputted.

  \item[documents]\mbox{}\\
  This directory to store the completed report. (not reference)

  \item[src]\mbox{}\\
  Directory where the \XeLaTeX source put. (require to be {\tt .tex} file extension)
\end{description}

\end{document}